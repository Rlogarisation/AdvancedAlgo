% Options for packages loaded elsewhere
\PassOptionsToPackage{unicode}{hyperref}
\PassOptionsToPackage{hyphens}{url}
%
\documentclass[
]{article}
\usepackage{amsmath,amssymb}
\usepackage{lmodern}
\usepackage{iftex}
\ifPDFTeX
  \usepackage[T1]{fontenc}
  \usepackage[utf8]{inputenc}
  \usepackage{textcomp} % provide euro and other symbols
\else % if luatex or xetex
  \usepackage{unicode-math}
  \defaultfontfeatures{Scale=MatchLowercase}
  \defaultfontfeatures[\rmfamily]{Ligatures=TeX,Scale=1}
\fi
% Use upquote if available, for straight quotes in verbatim environments
\IfFileExists{upquote.sty}{\usepackage{upquote}}{}
\IfFileExists{microtype.sty}{% use microtype if available
  \usepackage[]{microtype}
  \UseMicrotypeSet[protrusion]{basicmath} % disable protrusion for tt fonts
}{}
\makeatletter
\@ifundefined{KOMAClassName}{% if non-KOMA class
  \IfFileExists{parskip.sty}{%
    \usepackage{parskip}
  }{% else
    \setlength{\parindent}{0pt}
    \setlength{\parskip}{6pt plus 2pt minus 1pt}}
}{% if KOMA class
  \KOMAoptions{parskip=half}}
\makeatother
\usepackage{xcolor}
\IfFileExists{xurl.sty}{\usepackage{xurl}}{} % add URL line breaks if available
\IfFileExists{bookmark.sty}{\usepackage{bookmark}}{\usepackage{hyperref}}
\hypersetup{
  hidelinks,
  pdfcreator={LaTeX via pandoc}}
\urlstyle{same} % disable monospaced font for URLs
\usepackage{color}
\usepackage{fancyvrb}
\newcommand{\VerbBar}{|}
\newcommand{\VERB}{\Verb[commandchars=\\\{\}]}
\DefineVerbatimEnvironment{Highlighting}{Verbatim}{commandchars=\\\{\}}
% Add ',fontsize=\small' for more characters per line
\newenvironment{Shaded}{}{}
\newcommand{\AlertTok}[1]{\textcolor[rgb]{1.00,0.00,0.00}{\textbf{#1}}}
\newcommand{\AnnotationTok}[1]{\textcolor[rgb]{0.38,0.63,0.69}{\textbf{\textit{#1}}}}
\newcommand{\AttributeTok}[1]{\textcolor[rgb]{0.49,0.56,0.16}{#1}}
\newcommand{\BaseNTok}[1]{\textcolor[rgb]{0.25,0.63,0.44}{#1}}
\newcommand{\BuiltInTok}[1]{#1}
\newcommand{\CharTok}[1]{\textcolor[rgb]{0.25,0.44,0.63}{#1}}
\newcommand{\CommentTok}[1]{\textcolor[rgb]{0.38,0.63,0.69}{\textit{#1}}}
\newcommand{\CommentVarTok}[1]{\textcolor[rgb]{0.38,0.63,0.69}{\textbf{\textit{#1}}}}
\newcommand{\ConstantTok}[1]{\textcolor[rgb]{0.53,0.00,0.00}{#1}}
\newcommand{\ControlFlowTok}[1]{\textcolor[rgb]{0.00,0.44,0.13}{\textbf{#1}}}
\newcommand{\DataTypeTok}[1]{\textcolor[rgb]{0.56,0.13,0.00}{#1}}
\newcommand{\DecValTok}[1]{\textcolor[rgb]{0.25,0.63,0.44}{#1}}
\newcommand{\DocumentationTok}[1]{\textcolor[rgb]{0.73,0.13,0.13}{\textit{#1}}}
\newcommand{\ErrorTok}[1]{\textcolor[rgb]{1.00,0.00,0.00}{\textbf{#1}}}
\newcommand{\ExtensionTok}[1]{#1}
\newcommand{\FloatTok}[1]{\textcolor[rgb]{0.25,0.63,0.44}{#1}}
\newcommand{\FunctionTok}[1]{\textcolor[rgb]{0.02,0.16,0.49}{#1}}
\newcommand{\ImportTok}[1]{#1}
\newcommand{\InformationTok}[1]{\textcolor[rgb]{0.38,0.63,0.69}{\textbf{\textit{#1}}}}
\newcommand{\KeywordTok}[1]{\textcolor[rgb]{0.00,0.44,0.13}{\textbf{#1}}}
\newcommand{\NormalTok}[1]{#1}
\newcommand{\OperatorTok}[1]{\textcolor[rgb]{0.40,0.40,0.40}{#1}}
\newcommand{\OtherTok}[1]{\textcolor[rgb]{0.00,0.44,0.13}{#1}}
\newcommand{\PreprocessorTok}[1]{\textcolor[rgb]{0.74,0.48,0.00}{#1}}
\newcommand{\RegionMarkerTok}[1]{#1}
\newcommand{\SpecialCharTok}[1]{\textcolor[rgb]{0.25,0.44,0.63}{#1}}
\newcommand{\SpecialStringTok}[1]{\textcolor[rgb]{0.73,0.40,0.53}{#1}}
\newcommand{\StringTok}[1]{\textcolor[rgb]{0.25,0.44,0.63}{#1}}
\newcommand{\VariableTok}[1]{\textcolor[rgb]{0.10,0.09,0.49}{#1}}
\newcommand{\VerbatimStringTok}[1]{\textcolor[rgb]{0.25,0.44,0.63}{#1}}
\newcommand{\WarningTok}[1]{\textcolor[rgb]{0.38,0.63,0.69}{\textbf{\textit{#1}}}}
\setlength{\emergencystretch}{3em} % prevent overfull lines
\providecommand{\tightlist}{%
  \setlength{\itemsep}{0pt}\setlength{\parskip}{0pt}}
\setcounter{secnumdepth}{-\maxdimen} % remove section numbering
\ifLuaTeX
  \usepackage{selnolig}  % disable illegal ligatures
\fi

\author{}
\date{}

\begin{document}

In order to find the maximum number of files can be caught by the frog,
which can only jump forward from i to i+3 or i+4 pad. The status of this
question is position, which is the only factor is going to looped
through. The frog has two choices on each lily pad, they are jump
towards i+3 pad, or i+4 pad. Hence the maximum result can be generated
by combining the status and choices. At the end, dp{[}pos{]} can be
defined as maximum amount of files that can be caught by frog at current
position.

\begin{Shaded}
\begin{Highlighting}[]
\DataTypeTok{int} \FunctionTok{calculateMaxFiles}\OperatorTok{(}\DataTypeTok{int}\OperatorTok{[}\NormalTok{n}\OperatorTok{]}\NormalTok{ f}\OperatorTok{)} \OperatorTok{\{}

    \ControlFlowTok{return} \FunctionTok{dp}\OperatorTok{(}\DataTypeTok{int}\OperatorTok{[}\NormalTok{n}\OperatorTok{]}\NormalTok{ f}\OperatorTok{,} \DecValTok{0}\OperatorTok{);}

\OperatorTok{\}}

\DataTypeTok{int} \FunctionTok{dp}\OperatorTok{(}\DataTypeTok{int}\OperatorTok{[}\NormalTok{n}\OperatorTok{]}\NormalTok{ f}\OperatorTok{,} \DataTypeTok{int}\NormalTok{ pos}\OperatorTok{)} \OperatorTok{\{}

    \CommentTok{// Base case: End the loop if frog reaches end of lily pad.
}
    \ControlFlowTok{if} \OperatorTok{(}\NormalTok{pos }\OperatorTok{\textgreater{}=}\NormalTok{ EndofLilyPad}\OperatorTok{)} \OperatorTok{\{}

        \ControlFlowTok{return} \DecValTok{0}\OperatorTok{;}

    \OperatorTok{\}}

    \CommentTok{// Two choices: jump to i+3 or i+4.
}
    \ControlFlowTok{for} \OperatorTok{(}\DataTypeTok{int}\NormalTok{ i }\OperatorTok{=} \DecValTok{3}\OperatorTok{;}\NormalTok{ i }\OperatorTok{\textless{}=} \DecValTok{4}\OperatorTok{;}\NormalTok{ i}\OperatorTok{++)} \OperatorTok{\{}

        \DataTypeTok{int}\NormalTok{ maxInNextPosition }\OperatorTok{=} \FunctionTok{dp}\OperatorTok{(}\NormalTok{f}\OperatorTok{,}\NormalTok{ pos }\OperatorTok{+}\NormalTok{ i}\OperatorTok{);}

        \CommentTok{// Select the max between what we got so far,
}
        \CommentTok{// or future possible choices,
}
        \CommentTok{// through depth first search.
}
        \ControlFlowTok{return} \BuiltInTok{Math}\OperatorTok{.}\FunctionTok{max}\OperatorTok{(}\FunctionTok{dp}\OperatorTok{(}\NormalTok{f}\OperatorTok{,}\NormalTok{ pos}\OperatorTok{),} \OperatorTok{(}\NormalTok{maxInNextPosition }\OperatorTok{+}\NormalTok{ numOfFilesOnCurrentPad}\OperatorTok{);}

    \OperatorTok{\}}

\OperatorTok{\}}
\end{Highlighting}
\end{Shaded}

Overall the time complexity to solve this question is O(n\^{}2).

\end{document}

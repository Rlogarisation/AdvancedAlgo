% Options for packages loaded elsewhere
\PassOptionsToPackage{unicode}{hyperref}
\PassOptionsToPackage{hyphens}{url}
%
\documentclass[
]{article}
\usepackage{amsmath,amssymb}
\usepackage{lmodern}
\usepackage{iftex}
\ifPDFTeX
  \usepackage[T1]{fontenc}
  \usepackage[utf8]{inputenc}
  \usepackage{textcomp} % provide euro and other symbols
\else % if luatex or xetex
  \usepackage{unicode-math}
  \defaultfontfeatures{Scale=MatchLowercase}
  \defaultfontfeatures[\rmfamily]{Ligatures=TeX,Scale=1}
\fi
% Use upquote if available, for straight quotes in verbatim environments
\IfFileExists{upquote.sty}{\usepackage{upquote}}{}
\IfFileExists{microtype.sty}{% use microtype if available
  \usepackage[]{microtype}
  \UseMicrotypeSet[protrusion]{basicmath} % disable protrusion for tt fonts
}{}
\makeatletter
\@ifundefined{KOMAClassName}{% if non-KOMA class
  \IfFileExists{parskip.sty}{%
    \usepackage{parskip}
  }{% else
    \setlength{\parindent}{0pt}
    \setlength{\parskip}{6pt plus 2pt minus 1pt}}
}{% if KOMA class
  \KOMAoptions{parskip=half}}
\makeatother
\usepackage{xcolor}
\IfFileExists{xurl.sty}{\usepackage{xurl}}{} % add URL line breaks if available
\IfFileExists{bookmark.sty}{\usepackage{bookmark}}{\usepackage{hyperref}}
\hypersetup{
  hidelinks,
  pdfcreator={LaTeX via pandoc}}
\urlstyle{same} % disable monospaced font for URLs
\usepackage{color}
\usepackage{fancyvrb}
\newcommand{\VerbBar}{|}
\newcommand{\VERB}{\Verb[commandchars=\\\{\}]}
\DefineVerbatimEnvironment{Highlighting}{Verbatim}{commandchars=\\\{\}}
% Add ',fontsize=\small' for more characters per line
\newenvironment{Shaded}{}{}
\newcommand{\AlertTok}[1]{\textcolor[rgb]{1.00,0.00,0.00}{\textbf{#1}}}
\newcommand{\AnnotationTok}[1]{\textcolor[rgb]{0.38,0.63,0.69}{\textbf{\textit{#1}}}}
\newcommand{\AttributeTok}[1]{\textcolor[rgb]{0.49,0.56,0.16}{#1}}
\newcommand{\BaseNTok}[1]{\textcolor[rgb]{0.25,0.63,0.44}{#1}}
\newcommand{\BuiltInTok}[1]{#1}
\newcommand{\CharTok}[1]{\textcolor[rgb]{0.25,0.44,0.63}{#1}}
\newcommand{\CommentTok}[1]{\textcolor[rgb]{0.38,0.63,0.69}{\textit{#1}}}
\newcommand{\CommentVarTok}[1]{\textcolor[rgb]{0.38,0.63,0.69}{\textbf{\textit{#1}}}}
\newcommand{\ConstantTok}[1]{\textcolor[rgb]{0.53,0.00,0.00}{#1}}
\newcommand{\ControlFlowTok}[1]{\textcolor[rgb]{0.00,0.44,0.13}{\textbf{#1}}}
\newcommand{\DataTypeTok}[1]{\textcolor[rgb]{0.56,0.13,0.00}{#1}}
\newcommand{\DecValTok}[1]{\textcolor[rgb]{0.25,0.63,0.44}{#1}}
\newcommand{\DocumentationTok}[1]{\textcolor[rgb]{0.73,0.13,0.13}{\textit{#1}}}
\newcommand{\ErrorTok}[1]{\textcolor[rgb]{1.00,0.00,0.00}{\textbf{#1}}}
\newcommand{\ExtensionTok}[1]{#1}
\newcommand{\FloatTok}[1]{\textcolor[rgb]{0.25,0.63,0.44}{#1}}
\newcommand{\FunctionTok}[1]{\textcolor[rgb]{0.02,0.16,0.49}{#1}}
\newcommand{\ImportTok}[1]{#1}
\newcommand{\InformationTok}[1]{\textcolor[rgb]{0.38,0.63,0.69}{\textbf{\textit{#1}}}}
\newcommand{\KeywordTok}[1]{\textcolor[rgb]{0.00,0.44,0.13}{\textbf{#1}}}
\newcommand{\NormalTok}[1]{#1}
\newcommand{\OperatorTok}[1]{\textcolor[rgb]{0.40,0.40,0.40}{#1}}
\newcommand{\OtherTok}[1]{\textcolor[rgb]{0.00,0.44,0.13}{#1}}
\newcommand{\PreprocessorTok}[1]{\textcolor[rgb]{0.74,0.48,0.00}{#1}}
\newcommand{\RegionMarkerTok}[1]{#1}
\newcommand{\SpecialCharTok}[1]{\textcolor[rgb]{0.25,0.44,0.63}{#1}}
\newcommand{\SpecialStringTok}[1]{\textcolor[rgb]{0.73,0.40,0.53}{#1}}
\newcommand{\StringTok}[1]{\textcolor[rgb]{0.25,0.44,0.63}{#1}}
\newcommand{\VariableTok}[1]{\textcolor[rgb]{0.10,0.09,0.49}{#1}}
\newcommand{\VerbatimStringTok}[1]{\textcolor[rgb]{0.25,0.44,0.63}{#1}}
\newcommand{\WarningTok}[1]{\textcolor[rgb]{0.38,0.63,0.69}{\textbf{\textit{#1}}}}
\setlength{\emergencystretch}{3em} % prevent overfull lines
\providecommand{\tightlist}{%
  \setlength{\itemsep}{0pt}\setlength{\parskip}{0pt}}
\setcounter{secnumdepth}{-\maxdimen} % remove section numbering
\ifLuaTeX
  \usepackage{selnolig}  % disable illegal ligatures
\fi

\author{}
\date{}

\begin{document}

This question can be solved classic dynamic algorithm, which requires to
loop through all possible solution in order to obtain maximum amount of
enjoyment overall, however there is one extra restriction for this
question, that is same activities cannot be chosen for two days in a
row. The looping variable is based on the total number of days N as well
as activities, the choices are choosing between activity 1, 2, or 3. The
function dp{[}i{]} {[}j{]} can be defined as the maximum enjoyment by
playing activity j at day i.

\begin{Shaded}
\begin{Highlighting}[]
\DataTypeTok{int} \FunctionTok{calculateMaxEnjoyment}\OperatorTok{(}\DataTypeTok{int}\OperatorTok{[}\NormalTok{n}\OperatorTok{][}\DecValTok{3}\OperatorTok{]}\NormalTok{ e}\OperatorTok{)} \OperatorTok{\{}

    \DataTypeTok{int}\OperatorTok{[}\NormalTok{n}\OperatorTok{][}\DecValTok{3}\OperatorTok{]}\NormalTok{ dp }\OperatorTok{=} \DataTypeTok{int}\OperatorTok{[}\NormalTok{n}\OperatorTok{][}\DecValTok{3}\OperatorTok{];}

    \DataTypeTok{int}\NormalTok{ prevChoice }\OperatorTok{=} \DecValTok{0}\OperatorTok{;}

    \CommentTok{// Base case, when 0 days no activity yet.
}
\NormalTok{    dp}\OperatorTok{[}\DecValTok{0}\OperatorTok{][}\NormalTok{j}\OperatorTok{]} \OperatorTok{=} \DecValTok{0}\OperatorTok{;}

    

    \ControlFlowTok{for} \OperatorTok{(}\DataTypeTok{int}\NormalTok{ i }\OperatorTok{=} \DecValTok{1}\OperatorTok{;}\NormalTok{ i }\OperatorTok{\textless{}=}\NormalTok{ n}\OperatorTok{;}\NormalTok{ i}\OperatorTok{++)} \OperatorTok{\{}

        \ControlFlowTok{for} \OperatorTok{(}\DataTypeTok{int}\NormalTok{ j }\OperatorTok{=} \DecValTok{1}\OperatorTok{;}\NormalTok{ j }\OperatorTok{\textless{}=} \DecValTok{3}\OperatorTok{;}\NormalTok{ j}\OperatorTok{++)} \OperatorTok{\{}

            \DataTypeTok{int}\NormalTok{ activityEnjoy1 }\OperatorTok{=}\NormalTok{ dp}\OperatorTok{[}\NormalTok{i}\OperatorTok{{-}}\DecValTok{1}\OperatorTok{][}\DecValTok{1}\OperatorTok{]} \OperatorTok{+}\NormalTok{ e}\OperatorTok{[}\NormalTok{i}\OperatorTok{][}\NormalTok{j}\OperatorTok{];}

            \DataTypeTok{int}\NormalTok{ activityEnjoy2 }\OperatorTok{=}\NormalTok{ dp}\OperatorTok{[}\NormalTok{i}\OperatorTok{{-}}\DecValTok{1}\OperatorTok{][}\DecValTok{2}\OperatorTok{]} \OperatorTok{+}\NormalTok{ e}\OperatorTok{[}\NormalTok{i}\OperatorTok{][}\NormalTok{j}\OperatorTok{];}

            \DataTypeTok{int}\NormalTok{ activityEnjoy3 }\OperatorTok{=}\NormalTok{ dp}\OperatorTok{[}\NormalTok{i}\OperatorTok{{-}}\DecValTok{1}\OperatorTok{][}\DecValTok{3}\OperatorTok{]} \OperatorTok{+}\NormalTok{ e}\OperatorTok{[}\NormalTok{i}\OperatorTok{][}\NormalTok{j}\OperatorTok{];}

            \CommentTok{// Avaiable choices based on previous choice.
}
            \ControlFlowTok{if} \OperatorTok{(}\NormalTok{prevChoice }\OperatorTok{==} \DecValTok{1}\OperatorTok{)} \OperatorTok{\{}

                \CommentTok{// I can only choose activity 2 or 3 if I chose activity 1 yesterday.
}
\NormalTok{                dp}\OperatorTok{[}\NormalTok{i}\OperatorTok{][}\NormalTok{j}\OperatorTok{]} \OperatorTok{=} \BuiltInTok{Math}\OperatorTok{.}\FunctionTok{max}\OperatorTok{(}\NormalTok{activityEnjoy2}\OperatorTok{,}\NormalTok{ activityEnjoy3}\OperatorTok{);}

            \OperatorTok{\}}

            \ControlFlowTok{else} \ControlFlowTok{if} \OperatorTok{(}\NormalTok{prevChoice }\OperatorTok{==} \DecValTok{2}\OperatorTok{)} \OperatorTok{\{}

\NormalTok{                dp}\OperatorTok{[}\NormalTok{i}\OperatorTok{][}\NormalTok{j}\OperatorTok{]} \OperatorTok{=} \BuiltInTok{Math}\OperatorTok{.}\FunctionTok{max}\OperatorTok{(}\NormalTok{activityEnjoy1}\OperatorTok{,}\NormalTok{ activityEnjoy3}\OperatorTok{);}

            \OperatorTok{\}}

            \ControlFlowTok{else} \ControlFlowTok{if} \OperatorTok{(}\NormalTok{prevChoice }\OperatorTok{==} \DecValTok{3}\OperatorTok{)} \OperatorTok{\{}

\NormalTok{                dp}\OperatorTok{[}\NormalTok{i}\OperatorTok{][}\NormalTok{j}\OperatorTok{]} \OperatorTok{=} \BuiltInTok{Math}\OperatorTok{.}\FunctionTok{max}\OperatorTok{(}\NormalTok{activityEnjoy1}\OperatorTok{,}\NormalTok{ activityEnjoy2}\OperatorTok{);}

            \OperatorTok{\}}

            \ControlFlowTok{else} \OperatorTok{\{}

                \CommentTok{// There is no prev choice == 1st day.
}
                \CommentTok{// I can choose any activity I want.
}
\NormalTok{                dp}\OperatorTok{[}\NormalTok{i}\OperatorTok{][}\NormalTok{j}\OperatorTok{]} \OperatorTok{=} \BuiltInTok{Math}\OperatorTok{.}\FunctionTok{max}\OperatorTok{(}\NormalTok{activityEnjoy1}\OperatorTok{,}\NormalTok{ activityEnjoy2}\OperatorTok{,}\NormalTok{ activityEnjoy3}\OperatorTok{);}

            \OperatorTok{\}}

        \OperatorTok{\}}       

    \OperatorTok{\}}

    \ControlFlowTok{return} \BuiltInTok{Math}\OperatorTok{.}\FunctionTok{max}\OperatorTok{(}\NormalTok{dp}\OperatorTok{[}\NormalTok{n}\OperatorTok{][}\DecValTok{1}\OperatorTok{],}\NormalTok{ dp}\OperatorTok{[}\NormalTok{n}\OperatorTok{][}\DecValTok{2}\OperatorTok{],}\NormalTok{ dp}\OperatorTok{[}\NormalTok{n}\OperatorTok{][}\DecValTok{3}\OperatorTok{]);}

\OperatorTok{\}}

\end{Highlighting}
\end{Shaded}

Hence the function can calculate the maximal total enjoyment during the
trip, the actual pattern of choosing each trip can be obtained by
observing and recording the position of maximal dp value in last day,
then move one day backward, until the first day has been reached. The
overall time complexity for this algorithm is O(n\^{}2) since two loops
have been used to produce dp table.

\end{document}

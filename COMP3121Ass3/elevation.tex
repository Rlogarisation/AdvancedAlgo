% Options for packages loaded elsewhere
\PassOptionsToPackage{unicode}{hyperref}
\PassOptionsToPackage{hyphens}{url}
%
\documentclass[
]{article}
\usepackage{amsmath,amssymb}
\usepackage{lmodern}
\usepackage{iftex}
\ifPDFTeX
  \usepackage[T1]{fontenc}
  \usepackage[utf8]{inputenc}
  \usepackage{textcomp} % provide euro and other symbols
\else % if luatex or xetex
  \usepackage{unicode-math}
  \defaultfontfeatures{Scale=MatchLowercase}
  \defaultfontfeatures[\rmfamily]{Ligatures=TeX,Scale=1}
\fi
% Use upquote if available, for straight quotes in verbatim environments
\IfFileExists{upquote.sty}{\usepackage{upquote}}{}
\IfFileExists{microtype.sty}{% use microtype if available
  \usepackage[]{microtype}
  \UseMicrotypeSet[protrusion]{basicmath} % disable protrusion for tt fonts
}{}
\makeatletter
\@ifundefined{KOMAClassName}{% if non-KOMA class
  \IfFileExists{parskip.sty}{%
    \usepackage{parskip}
  }{% else
    \setlength{\parindent}{0pt}
    \setlength{\parskip}{6pt plus 2pt minus 1pt}}
}{% if KOMA class
  \KOMAoptions{parskip=half}}
\makeatother
\usepackage{xcolor}
\IfFileExists{xurl.sty}{\usepackage{xurl}}{} % add URL line breaks if available
\IfFileExists{bookmark.sty}{\usepackage{bookmark}}{\usepackage{hyperref}}
\hypersetup{
  hidelinks,
  pdfcreator={LaTeX via pandoc}}
\urlstyle{same} % disable monospaced font for URLs
\usepackage{color}
\usepackage{fancyvrb}
\newcommand{\VerbBar}{|}
\newcommand{\VERB}{\Verb[commandchars=\\\{\}]}
\DefineVerbatimEnvironment{Highlighting}{Verbatim}{commandchars=\\\{\}}
% Add ',fontsize=\small' for more characters per line
\newenvironment{Shaded}{}{}
\newcommand{\AlertTok}[1]{\textcolor[rgb]{1.00,0.00,0.00}{\textbf{#1}}}
\newcommand{\AnnotationTok}[1]{\textcolor[rgb]{0.38,0.63,0.69}{\textbf{\textit{#1}}}}
\newcommand{\AttributeTok}[1]{\textcolor[rgb]{0.49,0.56,0.16}{#1}}
\newcommand{\BaseNTok}[1]{\textcolor[rgb]{0.25,0.63,0.44}{#1}}
\newcommand{\BuiltInTok}[1]{#1}
\newcommand{\CharTok}[1]{\textcolor[rgb]{0.25,0.44,0.63}{#1}}
\newcommand{\CommentTok}[1]{\textcolor[rgb]{0.38,0.63,0.69}{\textit{#1}}}
\newcommand{\CommentVarTok}[1]{\textcolor[rgb]{0.38,0.63,0.69}{\textbf{\textit{#1}}}}
\newcommand{\ConstantTok}[1]{\textcolor[rgb]{0.53,0.00,0.00}{#1}}
\newcommand{\ControlFlowTok}[1]{\textcolor[rgb]{0.00,0.44,0.13}{\textbf{#1}}}
\newcommand{\DataTypeTok}[1]{\textcolor[rgb]{0.56,0.13,0.00}{#1}}
\newcommand{\DecValTok}[1]{\textcolor[rgb]{0.25,0.63,0.44}{#1}}
\newcommand{\DocumentationTok}[1]{\textcolor[rgb]{0.73,0.13,0.13}{\textit{#1}}}
\newcommand{\ErrorTok}[1]{\textcolor[rgb]{1.00,0.00,0.00}{\textbf{#1}}}
\newcommand{\ExtensionTok}[1]{#1}
\newcommand{\FloatTok}[1]{\textcolor[rgb]{0.25,0.63,0.44}{#1}}
\newcommand{\FunctionTok}[1]{\textcolor[rgb]{0.02,0.16,0.49}{#1}}
\newcommand{\ImportTok}[1]{#1}
\newcommand{\InformationTok}[1]{\textcolor[rgb]{0.38,0.63,0.69}{\textbf{\textit{#1}}}}
\newcommand{\KeywordTok}[1]{\textcolor[rgb]{0.00,0.44,0.13}{\textbf{#1}}}
\newcommand{\NormalTok}[1]{#1}
\newcommand{\OperatorTok}[1]{\textcolor[rgb]{0.40,0.40,0.40}{#1}}
\newcommand{\OtherTok}[1]{\textcolor[rgb]{0.00,0.44,0.13}{#1}}
\newcommand{\PreprocessorTok}[1]{\textcolor[rgb]{0.74,0.48,0.00}{#1}}
\newcommand{\RegionMarkerTok}[1]{#1}
\newcommand{\SpecialCharTok}[1]{\textcolor[rgb]{0.25,0.44,0.63}{#1}}
\newcommand{\SpecialStringTok}[1]{\textcolor[rgb]{0.73,0.40,0.53}{#1}}
\newcommand{\StringTok}[1]{\textcolor[rgb]{0.25,0.44,0.63}{#1}}
\newcommand{\VariableTok}[1]{\textcolor[rgb]{0.10,0.09,0.49}{#1}}
\newcommand{\VerbatimStringTok}[1]{\textcolor[rgb]{0.25,0.44,0.63}{#1}}
\newcommand{\WarningTok}[1]{\textcolor[rgb]{0.38,0.63,0.69}{\textbf{\textit{#1}}}}
\setlength{\emergencystretch}{3em} % prevent overfull lines
\providecommand{\tightlist}{%
  \setlength{\itemsep}{0pt}\setlength{\parskip}{0pt}}
\setcounter{secnumdepth}{-\maxdimen} % remove section numbering
\ifLuaTeX
  \usepackage{selnolig}  % disable illegal ligatures
\fi

\author{}
\date{}

\begin{document}

There are two status of this question, rows and columns, which required
to be looped through in order to find the best solution. There are two
choices as well, going right and going down.
\texttt{dp{[}row{]}{[}column{]}} can be defined as the minimum amount of
path with lower to higher elevation at current position. Hence this
function needs to return \texttt{dp{[}column{]}{[}0{]}} since where is
the destination as final result.

The way to find minimum number of moves from lower elevation to higher
elevation can be solved by the
\texttt{minElevatedPath(int{[}R{]}{[}C{]}\ elevation)} below.

\begin{Shaded}
\begin{Highlighting}[]
\DataTypeTok{int} \FunctionTok{minElevatedPath}\OperatorTok{(}\DataTypeTok{int}\OperatorTok{[}\NormalTok{R}\OperatorTok{][}\NormalTok{C}\OperatorTok{]}\NormalTok{ elevation}\OperatorTok{)} \OperatorTok{\{}

    \CommentTok{// dp has block shifted compared to elevation[R][C],
}
    \CommentTok{// hence all elevation coordination related need to +1.
}
    \DataTypeTok{int}\OperatorTok{[][]}\NormalTok{ dp }\OperatorTok{=} \KeywordTok{new} \DataTypeTok{int}\OperatorTok{[}\NormalTok{R}\OperatorTok{][}\NormalTok{C}\OperatorTok{];}

    \CommentTok{// Base case: starting point (0, row) has no elevation change yet.
}
\NormalTok{    dp}\OperatorTok{[}\DecValTok{0}\OperatorTok{][}\NormalTok{R}\OperatorTok{]} \OperatorTok{=} \DecValTok{0}\OperatorTok{;}

    \CommentTok{// Other base cases: calculate the min path elevation 
}
    \CommentTok{// for first row on top, 
}
    \CommentTok{// and first column on left as precondition.
}
    \ControlFlowTok{for} \OperatorTok{(}\DataTypeTok{int}\NormalTok{ i }\OperatorTok{=} \DecValTok{1}\OperatorTok{;}\NormalTok{ i }\OperatorTok{\textless{}}\NormalTok{ C}\OperatorTok{;}\NormalTok{ i}\OperatorTok{++)} \OperatorTok{\{}

\NormalTok{        dp}\OperatorTok{[}\NormalTok{i}\OperatorTok{][}\NormalTok{R}\OperatorTok{]} \OperatorTok{=}\NormalTok{ dp}\OperatorTok{[}\NormalTok{i}\OperatorTok{{-}}\DecValTok{1}\OperatorTok{][}\NormalTok{R}\OperatorTok{]} \OperatorTok{+} 

            \FunctionTok{elevationCal}\OperatorTok{(}\NormalTok{elevation}\OperatorTok{[}\NormalTok{i}\OperatorTok{{-}}\DecValTok{1}\OperatorTok{+}\DecValTok{1}\OperatorTok{][}\NormalTok{R}\OperatorTok{+}\DecValTok{1}\OperatorTok{],}\NormalTok{ elevation}\OperatorTok{[}\NormalTok{i}\OperatorTok{+}\DecValTok{1}\OperatorTok{][}\NormalTok{R}\OperatorTok{+}\DecValTok{1}\OperatorTok{]);}

    \OperatorTok{\}}

    \ControlFlowTok{for} \OperatorTok{(}\DataTypeTok{int}\NormalTok{ j }\OperatorTok{=}\NormalTok{ R}\OperatorTok{{-}}\DecValTok{1}\OperatorTok{;}\NormalTok{ j }\OperatorTok{\textgreater{}} \DecValTok{0}\OperatorTok{;}\NormalTok{ j}\OperatorTok{{-}{-})} \OperatorTok{\{}

\NormalTok{        dp}\OperatorTok{[}\NormalTok{C}\OperatorTok{][}\NormalTok{j}\OperatorTok{]} \OperatorTok{=}\NormalTok{ dp}\OperatorTok{[}\NormalTok{C}\OperatorTok{][}\NormalTok{j}\OperatorTok{{-}}\DecValTok{1}\OperatorTok{]} \OperatorTok{+} 

            \FunctionTok{elevationCal}\OperatorTok{(}\NormalTok{elevation}\OperatorTok{[}\NormalTok{C}\OperatorTok{+}\DecValTok{1}\OperatorTok{][}\NormalTok{j}\OperatorTok{{-}}\DecValTok{1}\OperatorTok{+}\DecValTok{1}\OperatorTok{],}\NormalTok{ elevation}\OperatorTok{[}\NormalTok{C}\OperatorTok{+}\DecValTok{1}\OperatorTok{][}\NormalTok{j}\OperatorTok{+}\DecValTok{1}\OperatorTok{]);}

    \OperatorTok{\}}

    \CommentTok{// Going right.
}
    \ControlFlowTok{for} \OperatorTok{(}\DataTypeTok{int}\NormalTok{ i }\OperatorTok{=} \DecValTok{1}\OperatorTok{;}\NormalTok{ i }\OperatorTok{\textless{}}\NormalTok{ C}\OperatorTok{;}\NormalTok{ i}\OperatorTok{++)} \OperatorTok{\{}

        \CommentTok{// Going down.
}
        \ControlFlowTok{for} \OperatorTok{(}\DataTypeTok{int}\NormalTok{ j }\OperatorTok{=}\NormalTok{ R}\OperatorTok{{-}}\DecValTok{1}\OperatorTok{;}\NormalTok{ j }\OperatorTok{\textgreater{}} \DecValTok{0}\OperatorTok{;}\NormalTok{ j}\OperatorTok{{-}{-})} \OperatorTok{\{}

\NormalTok{            dp}\OperatorTok{[}\NormalTok{i}\OperatorTok{][}\NormalTok{j}\OperatorTok{]} \OperatorTok{=} \BuiltInTok{Math}\OperatorTok{.}\FunctionTok{min}\OperatorTok{(}\NormalTok{dp}\OperatorTok{[}\NormalTok{i }\OperatorTok{+} \DecValTok{1}\OperatorTok{][}\NormalTok{j}\OperatorTok{]} \OperatorTok{+} 

                                \FunctionTok{elevationCal}\OperatorTok{(}\NormalTok{elevation}\OperatorTok{[}\NormalTok{i}\OperatorTok{+}\DecValTok{1}\OperatorTok{][}\NormalTok{j}\OperatorTok{+}\DecValTok{1}\OperatorTok{],} 

\NormalTok{                                             elevation}\OperatorTok{[}\NormalTok{i}\OperatorTok{+}\DecValTok{1}\OperatorTok{+}\DecValTok{1}\OperatorTok{][}\NormalTok{j}\OperatorTok{+}\DecValTok{1}\OperatorTok{]),} 

\NormalTok{                                dp}\OperatorTok{[}\NormalTok{i}\OperatorTok{][}\NormalTok{j }\OperatorTok{{-}} \DecValTok{1}\OperatorTok{]} \OperatorTok{+} 

                                \FunctionTok{elevationCal}\OperatorTok{(}\NormalTok{elevation}\OperatorTok{[}\NormalTok{i}\OperatorTok{+}\DecValTok{1}\OperatorTok{][}\NormalTok{j}\OperatorTok{+}\DecValTok{1}\OperatorTok{],} 

\NormalTok{                                             elevation}\OperatorTok{[}\NormalTok{i}\OperatorTok{+}\DecValTok{1}\OperatorTok{][}\NormalTok{j}\OperatorTok{{-}}\DecValTok{1}\OperatorTok{+}\DecValTok{1}\OperatorTok{]));}

        \OperatorTok{\}}

    \OperatorTok{\}}

    \ControlFlowTok{return}\NormalTok{ dp}\OperatorTok{[}\NormalTok{column}\OperatorTok{][}\DecValTok{0}\OperatorTok{];}

\OperatorTok{\}}



\DataTypeTok{int} \FunctionTok{elevationCal}\OperatorTok{(}\DataTypeTok{int}\NormalTok{ heightA}\OperatorTok{,} \DataTypeTok{int}\NormalTok{ heightB}\OperatorTok{)} \OperatorTok{\{}

    \ControlFlowTok{if} \OperatorTok{(}\NormalTok{heightA }\OperatorTok{\textless{}}\NormalTok{ heightB}\OperatorTok{)} \OperatorTok{\{}\ControlFlowTok{return} \DecValTok{1}\OperatorTok{;\}} \ControlFlowTok{else} \OperatorTok{\{}\ControlFlowTok{return} \DecValTok{0}\OperatorTok{;\}}

\OperatorTok{\}}
\end{Highlighting}
\end{Shaded}

Also, the minimum actual path can be found by selecting the smaller dp
value for either top or left block in ending block, then move to the
selected block, repeat the process until reach the starting block. The
overall time complexity for this solution is O(n\^{}2).

\end{document}
